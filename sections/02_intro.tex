\begin{section}{Introduction}
    Experience shows an imersed object in water may have motion vertically: it can float up, remain 
    at its initial position or sink down. In the next sections we cover this phenomenology 
    \textbf{(II)} and explain it in terms of archimedes' principle \textbf{(III)}; then using a 
    simplified model, we show how this behavior is related to the object's density compared with the 
    density of water. Finally \textbf{(IV)} we apply this model to fish having swim bladder which use 
    in a very efficient way this phenomenology as a static lift method for buoyancy.
\end{section}
