\begin{section}{Conclusions}
    Many fish have as a mechanism for buoyancy the mass input/output in their swim baldder. 
    Adding mass from a critical value give this kind of fish the ability to sink down,
    mantaining mass in this critical value keep them floating at a fixed depth and lowering 
    the mass from this critical value is equivalent to add a negative mass amount which 
    makes fish become less dense than water so they begin to float up; this is a static lift 
    method since fish don't have to use their fins or drag forces from currents to generate 
    vertical motion in water. Then buoyancy for fish with swim bladder can be modelled with 
    just one parameter as shown in section \textbf{(IV)}, namely, $\rho_b$  which is measured
    from a critical value $\rho_0$ as reference so it can be possitive, zero or negative
    meaning this that fish are denser than water, have the same density of water and are less 
    dense than water, respectively.
\end{section}
