\begin{section}{Phenomenology}
    
    When an object is dropped in a swimming pool, there are three possibilities :

    \begin{enumerate}
        \item The object floats in the pool surface.
        \item The object dips into the water and stops at some point without reaching the bottom.
        \item The object dips until reaching the pool's bottom.
    \end{enumerate}
    
    When the object is dropped, it gains kinetic energy before dipping into the pool. In the first 
    case, the motion stops and the object begins to float until reaching the surface; then because 
    of  Newton's second law: \textbf{there is a force} pulling up the object which is greater than 
    the force of gravity and is balanced when the object is partially inmersed. In the second case 
    when the energy is dissipated because of water friction forces, the object stops and gets stuck 
    at the reached depth; then according to Newton's second law, \textbf{there is a force} with the 
    same magnitude and opposed to the force of gravity. Finally, in case three, when the object dips 
    into the water it begins to slow its motion (until it is stopped by the bottom), then again 
    because of Newton's second law : \textbf{there is a force} which partially balances the force of 
    gravity.
    
    The unknown force is the buoyancy force which was first described by Archimedes of Syracuse about
    200 B.C. Archimedes captured this phenomena in the well known Archimedes' principle which is 
    described in the next section.
\end{section}
